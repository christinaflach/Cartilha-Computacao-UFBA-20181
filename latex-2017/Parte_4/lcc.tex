\section{Licenciatura em Computação}

\subsection{Introdução}
O curso de Licenciatura em Computação(LC)\index{LC}, que teve suas atividades iniciadas em 2010, é oferecido pelo Instituto de Matemática e Estatística(IME) e segue as Diretrizes Curriculares Nacionais para a formação de professores da Educação Básica em Nível Superior, além de acompanhar as constantes transformações nos âmbitos científicos, educacionais e tecnológicos. 

A proposta do curso foi elaborada pela comissão escolhida pelo Departamento de Ciência da Computação, composta pelas professoras Anna Friedericka Schwarzelmüller, Débora Abdalla Santos e Laís Nascimento Salvador. As professoras contaram com a colaboração dos professores Aline Maria Santos Andrade, Celso Alberto Saibel Santos, Daniela Barreiro Claro e Manoel Gomes de Mendonça Neto, para suporte nas disciplinas com conteúdos da área tecnológica principalmente.

\subsection{Colegiado do Curso}
\begin{itemize}
	\item {Coordenadora: Anna Friedericka Schwarzelmuller}
    \item {Vice-coordenador: Ecivaldo Matos}
    \item {Sala do colegiado: 116 do Instituto de Matemática}
    \item {E-mail para contato: lc@ufba.br}
\end{itemize}

\subsection{Objetivos do Curso}
%\paragraph{} %Nao utilizar esse comando
    O site do curso de LC da UFBA define:
        \begin{quote}
        {\small De maneira geral pode-se estabelecer como objetivo do curso, formar profissionais de educação que atuem como agentes integradores no processo de ensino-aprendizagem, capazes de compreender o fenômeno educativo na sua diversidade e complexidade, contextualizando-o socialmente no seu tempo e    espaço.}
        \end{quote}

Vale acrescentar diversos outros objetivos. Primeiramente têm-se a finalidade de fornecer uma formação bem estruturada e sólida na compreensão dos problemas que envolvem a área de Ensino e inserção da tecnologia nesse meio.

Também, em uma área em constante mudança e aprimoramento, o curso traz uma proposta de gerar inovações durante a formação dos futuros educadores. Esses poderão exercer o magistério e estarão preparados para o mercado de trabalho.

Por fim, propõem-se a incentivar o aluno a programas de pós-graduação e o seu espírito científico.

\subsection{Perfil do Egresso}

O formando sai do curso preparado para atuar no magistério em nível de ensino básico, técnico e tecnológico, tendo qualificações pedagógicas e científicas. Espera-se que o egresso tenha domínio em aspectos básicos de ciência da computação e na área de educação, sendo capaz de realizar projetos interdisciplinares com outros docentes, e utilizar de tecnologias digitais para mudar positivamente o processo de aprendizado do estudante. 

Além disso, terá formação para ingressar em pós-graduação e em programas de ciência da computação ou outras áreas do gênero.

\subsection{Campos de Atuação}

O profissional formado em LC pode atuar como docente em ensino fundamental, médio e escolas técnicas, trabalhar com treinamento e qualificação em corporações, como consultor em empresas e instituições, e consultor técnico. Outras opções são empreendedorismo, avaliando e desenvolvendo softwares educacionais e de atividades de pesquisa na área de informática.

\subsection{Dados Gerais do Curso}
    \paragraph{Tempo de duração}
    \begin{quote}
	    \begin{itemize}
		    \item{Duração Mínima: 3.5 anos}
            \item{Duração Máxima: 7.5 anos}
	    \end{itemize}  
    \end{quote}

    \paragraph{Carga Horária}
    \begin{quote}
	    \begin{itemize}
    	    \item{Carga Horária Obrigatória: 2499 horas}
            \item{Carga Horária Optativa: 442 horas}
            \item{Atividade Complementar: 200 horas}
        \end{itemize}
    \end{quote}

\subsection{Base Curricular}
    \paragraph{}A grade curricular de LC possui foco em matérias de computação, matemática e especialmente pedagogia, que são um grande diferencial quando se compara esse curso com o curso de Bacharelado em Ciência da Computação e Sistema de Informação. Matérias como Filosofia da Educação e Fundamentos Psicológicos da Educação são exemplos que formam a base para profissionais da área de pedagogia. 


\begin{longtable}{K{1.7cm}K{6cm}K{1.4cm}K{0.9cm}K{1.5cm}}
\hline
\multicolumn{5}{c}{\textbf{Licenciatura em Computação}}\\
\hline
\multicolumn{5}{c}{\textbf{1$^o$ período}}\\
\hline
 Código & Nome da disciplina & (T-P-E) & C.H. & Requisitos\\
 \hline
EDCB80 & Filosofia da Educação & (2-2-0) & 68 & \\
MATA01 & Cálculo A & (4-0-0) & 68 \\
MATA37 & Introdução à Lógica de Programação & (2-2-0) & 68 & \\
MATA39 & Seminários de Introdução ao Curso & (3-0-0) & 51 \\
MATA42 & Matemática Discreta I & (4-0-0) & 68 \\
 \hline
\multicolumn{2}{c}{TOTAL} & 19 & 323\\
 \hline
 \multicolumn{2}{c}{TOTAL ACUMULADO} & 19 & 323\\
 
 \multicolumn{5}{c}{\textbf{2$^o$ período}}\\
\hline
 Código & Nome da disciplina & (T-P-E) & C.H. & Requisitos\\
 \hline
EDC287 & Educação e Tecnologias Contemporâneas & (2-2-0) & 68\\
MATD04 & Estrutura de Dados & (2-2-0) & 68 & MATA37\\
MATC81 & Sistemas Básicos de Computação:
Arquitetura e Software & (2-2-0) & 68 & MATA39\\
MATA97 & Introdução à Lógica Matemática & (2-2-0) & 68 & MATA42\\
MATA68 & Computador, Ética e Sociedade & (2-1-0) & 51\\
 \hline
\multicolumn{2}{c}{TOTAL} & 19 & 323\\
 \hline
 \multicolumn{2}{c}{TOTAL ACUMULADO} & 38 & 646\\
 \hline
 
 \multicolumn{5}{c}{\textbf{3$^o$ período}}\\
\hline
 Código & Nome da disciplina & (T-P-E) & C.H. & Requisitos\\
 \hline
EDCA01 & Fundamentos Psicológicos da Educação & (2-2-0) & 68 & \\
MATA55 & Programação Orientada a Objetos  & (2-2-0) & 68 & MATD01 \\
MAT236 & Métodos Estatísticos & (4-0-0) & 68 & MATA01 MATA42 \\
MATC74 & Introdução a Linguagens Formais e Autômatos & (2-2-0) & 68 & MATA42\\
OPT01 & --- & --- & 51 \\
 \hline
\multicolumn{2}{c}{TOTAL} & 19 & 323\\
 \hline
 \multicolumn{2}{c}{TOTAL ACUMULADO} & 57 & 969\\
 \hline
 
 \multicolumn{5}{c}{\textbf{4$^o$ período}}\\
\hline
 Código & Nome da disciplina & (T-P-E) & C.H. & Requisitos\\
 \hline
EDCA11 & Didática e Práxis Pedagógica I & (0-4-0) & 68 & \\
MATA62 & Engenharia de Software I & (2-2-0) & 68 & MATA55 \\
MATC82 & Sistemas Web & (2-2-0) & 68 & MATA55\\
MATA59 & Redes de Computadores I & (2-2-0) & 68 & MATC81\\
MATA41 & Informática na Educação & (2-2-0) & 68 \\
 \hline
\multicolumn{2}{c}{TOTAL} & 20 & 340\\
 \hline
 \multicolumn{2}{c}{TOTAL ACUMULADO} & 77 & 1309\\
 \hline
 
 \multicolumn{5}{c}{\textbf{5$^o$ período}}\\
\hline
 Código & Nome da disciplina & (T-P-E) & C.H. & Requisitos\\
 \hline
EDCA12 & Didáticas e Práxis Pedagógicas II & (0-4-0) & 68 & EDCA11 \\
MATB19 & Sistemas Multimídias  & (2-2-0) & 68 & MATA55 \\
MATD05 & Banco de Dados e Aplicações & (2-2-0) & 68 & MATD04 \\
MATB21 & Ambientes Interativos de Aprendizagem & (4-0-0) & 68 & MATA37 MATA41\\
EDC286 & Avaliação de Aprendizagem & (2-2-0) & 68 \\
 \hline
\multicolumn{2}{c}{TOTAL} & 20 & 340\\
 \hline
 \multicolumn{2}{c}{TOTAL ACUMULADO} & 97 & 1649\\
 \hline
 \newpage
 \hline
 
  \multicolumn{5}{c}{\textbf{6$^o$ período}}\\
\hline
 Código & Nome da disciplina & (T-P-E) & C.H. & Requisitos\\
 \hline
MATC68 & Estágio Supervisionado I & (0-0-4) & 68 & EDCA11 EDCA12\\
MATC72 & Interação Humano-Computador & (2-2-0) & 68 & MATB19 \\
MATC78 & Projeto de Software Educativo & (2-2-0 & 68 & MATA62\\
MATB22 & Laboratório de Informática na Educação & (0-3-0) & 51 & MATB21\\
MATC76 & Prática de Ensino de Computação I & (0-4-0) & 68 & EDCA11\\
 \hline
\multicolumn{2}{c}{TOTAL} & 19 & 323\\
 \hline
 \multicolumn{2}{c}{TOTAL ACUMULADO} & 116 & 1972 \\
 \hline
 
  \multicolumn{5}{c}{\textbf{7$^o$ período}}\\
\hline
 Código & Nome da disciplina & (T-P-E) & C.H. & Requisitos\\
 \hline
MATC69 & Estágio Supervisionado II & (0-0-4) & 68 & MATC68\\
MATC79 & Projetos Interdisciplinares: concepçao e ética  & (1-3-0) & 68 & EDCA11 MATA41 \\
MATB20 & Inteligência Artificial em Educação & (2-2-0) & 68 & MATA41, MATC73 \\
MATC77 & Prática de Ensino de Computação & (0-4-0) & 68 & MATA41 MATC73\\
OPT02 & --- & --- & 51 \\
 \hline
\multicolumn{2}{c}{TOTAL} & 19 & 323\\
 \hline
 \multicolumn{2}{c}{TOTAL ACUMULADO} & 135 & 2295\\
 \hline
 
   \multicolumn{5}{c}{\textbf{8$^o$ período}}\\
\hline
 Código & Nome da disciplina & (T-P-E) & C.H. & Requisitos\\
 \hline
MATC70 & Estágio Supervisionado III & (0-0-6) & 102 & MATC69\\
LETE46 & Libras-Língua Brasileira de Sinais & (1-1-0) & 34\\
OPT03 & --- & --- & 68\\
OPT04 & --- & --- & 51\\
OPT05 & --- & --- & 51\\
 \hline
\multicolumn{2}{c}{TOTAL} & 18 & 306\\
 \hline
 \multicolumn{2}{c}{TOTAL ACUMULADO} & 153 & 2601\\
 \hline
 
 
   \multicolumn{5}{c}{\textbf{9$^o$ período}}\\
\hline
 Código & Nome da disciplina & (T-P-E) & C.H. & Requisitos\\
 \hline
MATC69 & Estágio Supervisionado IV & (0-0-10) & 170 & MATC70\\
OPT06 & --- & --- & 68 \\
OPT07 & --- & --- & 51 \\
OPT08 & --- & --- & 51 \\
 \hline
\multicolumn{2}{c}{TOTAL} & 20 & 340\\
 \hline
 \multicolumn{2}{c}{TOTAL ACUMULADO} & 173 & 2941\\
 \hline
\end{longtable}

Legenda:\\
{\bf C.H.} Carga Horária\\
{\bf T} Carga horária de aula teórica\\
{\bf P} Carga horária de aula prática\\
{\bf E} Carga horária de estágio\\

