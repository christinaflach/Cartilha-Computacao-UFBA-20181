\chapter{Solicitações acadêmicas}
\DoPToC
	\section{Secretaria geral dos cursos}
	   \begin{wrapfigure}{R}{0.3\textwidth}
            \centering
            \includegraphics[width=0.29\textwidth]{sgc}
        \end{wrapfigure}
            
A Secretaria Geral de Cursos (SGC) Possui um Núcleo de Atendimento ao Estudante (NAE),que atende aos alunos da UFBA e à comunidade externa. O horário de atendimento do SGC é de segunda a sexta-feira, das 8h30min às 12h, e o Núcleo de Atendimento ao Estudante (NAE) funciona das 7h30min até as 18h.
               
O SGC se divide em algumas seções que possum propósitos específicos

\begin{itemize}
    \item SAE - Serviço de atendimento ao estudante
    \item SEARE-GRAD - Admissão e Registros Escolares da Graduação
    \item SEARE-PÓS - Admissão e Registros Escolares da Pós-Graduação
    \item SECUP - Currículos e Programas
    \item SEDIC - Diplomas e Certificados
    \item SEADI - Apoio Administrativo
    \item Arquivo - Responsável pela guarda de documentos acadêmicos
\end{itemize}


\section{Aproveitamento de estudo}
     \begin{wrapfigure}{R}{0.4\textwidth}
            \centering
            \includegraphics[width=0.39\textwidth]{historico}
        \end{wrapfigure}
            Poderão ser aproveitados estudos/atividades realizados na UFBA ou em outra instituição de ensino superior desde que requerido pelo interessado e instruídos com os seguintes documentos:
          \begin{itemize}
              \item Histórico escolar atualizado
                    \item Programas dos componentes curriculares cursados
                    \item Base legal que regulamenta o curso de origem
                \end{itemize}

O SECUP é o responsável para realizar esta avaliação curricular

\section{Alterações no curso}

O aluno dispõe de certos direitos e ferramentas relativos ao seu curso que também podem ser solicitados através do SGC, dentre elas estão:
                    
\subsection{Desistência definitiva de curso} 
Caso você opte por desistir do curso é necessário que você preencha o formulário de requerimento justificando sua decisão e entregue ao SUPAC.
                    
\subsection{Dilatação do prazo máximo para conclusão do curso} 
Caso você esteja prestes a exceder o prazo máximo de conclusão do seu curso, você pode o dilatar ainda o equivalente a metade do tempo máximo do seu curso.
                     
\subsection{Permanência no curso} 
Caso você tenha trancado e você deseja continuar, você faz um pedido de permanência do curso justificando o motivos que te levaram a trancar e porque você quer voltar.

\subsection{Revalidação de Diploma ou certificado} 
Caso você queira que seu diploma seja reconhecido internacionalmente, você busca por essa solicitação.

\section{Reavaliações}
\begin{wrapfigure}{R}{0.4\textwidth}
    \centering
    \includegraphics[width=0.39\textwidth]{reavaliacoes}
\end{wrapfigure}

\subsection{Retificação de Histórico}
É o procedimento que se deve adotar sempre que o histórico escolar não esteja correspondendo à realidade. Isento de taxa. Pode ser solicitada em qualquer época do ano. São as situações mais frequentes:

\subsection{ Ausência ou erro de nota/resultado}
Caso você ache injusta uma nota atribuída por um professor, você pode pedir que o professor reveja sua prova em um prazo de 72 horas, ou você pode recorrer ao colegiado para que seja reavaliado por estes.
                    
\subsection{ Ausência de disciplina}
Em caso de erro do sistema, você perca o registro de uma disciplina cursada, você pode pedir com que revejam seu histórico para que você venha a possuir sua disciplina novamente.

\section{Transferências}
O aluno também pode solicitar uma transferência para outro curso ou para outra universidade.

\subsection{Transferência ``ex-ofício''}

Caso, você se mude para outro estado, você pode solicitar com antecedência para que te transfiram para outra universidade federal.

\subsection{Transferência interna de caráter especial}
Quando você transfere de um curso para outro dentro da própria faculdade, você busca essa solicitação.
